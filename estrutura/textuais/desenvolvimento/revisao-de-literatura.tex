% REVISÃO DE LITERATURA--------------------------------------------------------

\chapter{\larger Revisão Literária}
\label{chap:fundamentacaoTeorica}

É uma boa prática iniciar cada novo capítulo com um breve texto introdutório (tipicamente, dois ou três parágrafos) que deve deixar claro o quê será discutido no capítulo, bem como a organização do capítulo.
Também servirá ao propósito de "amarrar"{} o conteúdo deste capítulo com o conteúdo do capítulo imediatamente anterior.

\section{A programação sob o viés matemático}
Esboço de um resumo dos dois primeiros capítulos do livro de otimização, o que vai justificar muita coisa na verdade temos que ver depois v
\subsection{Otimizando à uma variável}
\subsection{Otimizando à mais de uma variável}

%subscheqchion

\section{As primeiras três décadas da programação}
(1930-1960)
Dantzing, M, até -SQP.
\section{Programação não linear - O surgimento}
\section{Os pilares arquitetônicos do SCP}
\section{O modernismo sob o ponto de vista da programação (Uma ideia do o quê que a gente tem de moderno)}
