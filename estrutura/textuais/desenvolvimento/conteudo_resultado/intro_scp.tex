\section{Introdução ao SCP}

\noindent
O \textit{Sequential Cutting Plane} (SCP) \cite{Still2010} pode ser visto como uma demonstração da
capacidade das técnicas de otimização por problemas lineares que, como já dito, haviam
sido deixadas de lado. O método apresentado por Claus Still e Tapio Westerlund em 2010
pode ser visto como a conclusão do estudo deles sobre abordagens de subproblemas lineares
em problemas não lineares. Um estudo de pelo menos 5 anos, com o primeiro artigo sendo
uma apresentação do método em um estagio inicial aplicado a problemas de otimização
não linear inteira mista (MINLP) \cite{Still_2005}.

O método foi primariamente aplicado em problemas MINLP como parte de um algoritmo,
demonstrando melhor performance em problemas MINLP convexos. Outra possibilidade de
aplicação é o uso para resolução de problemas não lineares de grande escala, de grande
dimensionalidade, visto a problemática enfrentada por métodos quadráticos para tais
problemas.

\subsection{Aplicação SCP em MINLP}

\noindent
Uma das aplicações do SCP foi como parte de um otimizador de outra classe de problemas, na
otimização não linear inteira mista (MINLP). Existem problemas que buscam o ótimo usando
variáveis tanto discretas como continuas, com relações entre elas que são não lineares e
que também contém funções não convexas. Tudo isso dificulta muito a computação da solução.
Otimizadores capazes de resolverem esses tipos de problemas são ditos como os mais flexíveis,
já que seu escopo é tão abrangente. Isso o torna o centro das atenções por pesquisadores de
todas as áreas.

Problemas MINLP, na sua forma mais abstrata se dão da seguinte forma:

\vspace{-15pt}
\begin{mini!}
{x}{ f(x) \label{minlp_obj}}{\label{prob_minlp}}{}
\addConstraint{x}{\in \mathcal{F} \subseteq \mathbb{R}^n}{\label{r1_minlp}}
\end{mini!}
onde o conjunto \( \mathcal{F} \) pode ser tanto não linear quanto discreto, e \( f: \mathbb{R}^n \mapsto \mathbb{R}\) .
As diferentes escolhas
de \(f\) e \(\mathcal{F}\) levam as diferentes subclasses de MINLP, das quais temos algumas
denominações mais comuns, como:

\begin{itemize}
\item Otimização Convexa, onde \( \mathcal{F} \) é convexo, e o ótimo é inteiro;
\item Otimização Disjuntiva, onde algumas variáveis são continuas e outras são booleanas;
\item Otimização Não Linear, onde restrições não lineares constroem \(\mathcal{F}\) e/ou \(f\) é não linear. Normalmente usados métodos \textit{Sequential Quadratic Programming} (SQP);
\item Otimização por meio de grafos de expressões;
\item Otimização por meio de subespaços convexos e/ou linearizações de \(\mathcal{F}\).
\end{itemize}

Podemos encontrar que as formas mais gerais de MINLP são incomputáveis, visto que pode não existir
informações ou ferramentas para serem usadas na resolução do problema. Para problemas MINLP convexos,
e com restrições de desigualdades, a resolução de subproblemas lineares oferecem limites inferiores para
o ótimo, o que para o SCP, saber desses limites ajuda em questões de eficiência. Segundo os autores do
SCP, ao ser usado para problemas MINLP, foram encontradas soluções melhores em apenas um minuto do que
usando outros métodos que encontraram boas soluções em 12 horas. Essa implementação foi de uma versão
prévia do SCP, apresentada em 2006 \cite{Still_2005}, mas também com um código de difícil acessível.

A resolução de problemas MINLP convexos é importante no contexto global de problemas MINLP, já
que a maioria dos problemas são resolvidos por uma sequência de problemas que foram convertidos em
problemas convexos. Experiências numéricas mostram que o SCP pode ser aplicado para problemas MINLP
não convexos como solucionador de subproblemas. A convergência ao ponto estacionário, quando aplicado
a estes tipos de problemas, aparenta ser rápida o suficiente mesmo que o estado do algoritmo esteja
longe do ponto estacionário.