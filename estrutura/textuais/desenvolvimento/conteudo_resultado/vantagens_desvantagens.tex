\section{As vantagens e desvantagens computacionais do SCP}

\noindent
Agora pensando exclusivamente na implementação do método, primeiro foi necessária
a construção de uma estrutura para o armazenamento do problema. Essa estrutura guarda
a função objetivo do problema, uma lista de funções das restrições de desigualdades e
outra lista de funções para as restrições de igualdades. Para o armazenamento das
funções basta considerar ponteiros para elas. Outras informações que podem ser
armazenadas nessa estrutura do problema são as regiões de confiança, como os
pontos que de fato são. Para essa estrutura também foi criada uma função
para avaliar todos os aspectos básicos do problema em um certo ponto. Nessas
funções são avaliada a função objetivo e seu gradiente e todas as restrições e seus
gradientes. Essa função é usada em quase todas as vezes que se é requerida uma avaliação
do problema, não sendo usada em casos onde se usam certas funções isoladamente. A
possibilidade de se ter essa estrutura onde todas as funções podem ser avaliadas de uma vez
nos dá uma vantagem em questões de performance, uma vez que podemos otimizar o tempo de
realizar todas essas avaliações aumentando consideravelmente o tempo de execução do algoritmo.

Um outro efeito benéfico sobre o método ter esses momentos onde todas as funções são
avaliadas é um custo reduzido em problemas com funções muito difíceis. No entanto, ainda
é feita uma grande quantidade de avaliações, por isso, mesmo considerando esse efeito
benéfico, os autores não recomendam para problemas com funções difíceis.

O uso de subproblemas lineares é o ponto principal do método, e com ele vem todos
os efeitos. O problema linear, mesmo que grande, é de rápida computação e, a depender
do método de resolução, pode ser resolvido com operações simples da CPU, o que reafirma
a velocidade do método. No entanto, complexidade de processo não é a única preocupação
que se deve ter com um método capaz de resolver problemas de grande escala. Para que
possamos usar métodos já conhecidos para otimização linear, foi necessária a transformação
do problema no formato do artigo (\ref{prob_scpl}) em um equivalente em um formato mais
simples (\ref{prob_scplt}) de ser percebido e partir dele extraído a matriz
e os vetores (\ref{prob_linear_pdr}), e ai que encontramos o problema.

Mesmo sendo apenas números, o armazenamento da matriz e dos vetores pode vir a
se tornar um problema. Primeiro temos que considerar os tamanhos. Para o vetor
\(c\), temos que seu número de elementos é \(n+m_i+m_e+m_e\), o que de cara
não parece ser um problema, principalmente pelo fato de que se foi possível
armazenar as funções e restrições na estrutura do problema, deve-se ter
capacidade para armazenar as entradas de \(c\). No entanto, isso vale
quando consideramos que \(c \in \mathbb{R}^n\), mas se por exemplo, quisermos
adaptar o método para funcionar com números hiper-duais (\ref{sec_hiper_dual}),
que equivalem a quatro números reais, já existe um aumento considerável nos
requerimentos de armazenamento.

Sabendo quantos elementos tem o vetor \(c\), sabemos também quantas colunas tem
a matriz \(A\), visto que devem ser iguais para que possa ter \(Ax\). O que nos
falta agora é saber o número de linhas de \(A\), que também é o número de elemento
de \(b\), também para que se possa ter \(Ax = b\). Ao observarmos o problema
ajustado (\ref{prob_scplt}), podemos ver que o número de linhas é de pelo menos
\( m_i + m_e + m_e + n + n + m_i + m_i + m_e + m_e + m_e + m_e = 2n + 3m_i + 6m_e\).
É no mínimo isso pelo fato das restrições (\ref{r4_scplt}) e (\ref{r5_scplt})
adicionarem \(2n\) linhas a cada iterações de resolução de problemas lineares,
e se chegar a ter \(n\) iterações, são \(2n^2\) linhas adicionais.
Portanto, a matriz \(A\) tem no mínimo \( (2n + 3m_i+6m_e) \cdot  (n+m_i+2m_e) \)
elementos e no máximo  \( (2n^2 + 2n + 3m_i+6m_e) \cdot (n+m_i+2m_e)\), o que
pode facilmente se tornar uma dificuldade de uso em casos onde seria interessante
se ter os benefícios, um exemplo seria um dispositivo embarcado que resolva problemas
ideais para o SCP, porém tem pouca memória.

Outro ponto que podemos observar é a construção do ferramental para o funcionamento
do SCP. Mesmo que isso não seja relacionado ao funcionamento do método propriamente
dito, a existência do ferramental por si só já é uma vantagem. Isso talvez seja resultado
da posição histórica do método, uma vez que os autores foram capazes de estudar e
compreender as ferramentas que utilizaram em um estado muito mais refinado do que
se tinha, por exemplo, na década de 70.

Contanto só com as ferramentas mais básicas, como as operações entre vetores, números,
matrizes e cálculos de derivadas, podemos facilmente implementar métodos básicos, como
o método de Newton, e o gradiente descendente. Para o método de Newton, que requer o
uso das primeiras e segundas derivadas, temos duas opções. A primeira sendo calcular
a matriz Hessiana diretamente, o que é custoso, mas existindo a segunda opção para
cobrir o caso que é necessário mais desempenho, não é um problema. A segunda opção
é caracterizada pelo uso do BFGS (\ref{sec_bfgs}), podendo ser adaptado para calcular
o inverso da Hessiana, já que é o inverso que é usado no método de Newton. Além disso,
já existindo o cálculo pelo BFGS, podemos usar o próprio método de minimização do BFGS.
Para o método do gradiente descendente, pequenas mudanças devem ser feitas ao método de
busca em linha para que se possa escolher os limites para o tamanho do passo, uma
vez que para o SCP, na função Lagrangiana penalizada, o tamanho do passo é limitado
entre \(0\) e \(1\).

Também existem métodos mais complexos que podemos montar a partir do que foi construído
para o SCP. Foram construídas funções Lagrangianas, com e sem penalidades, e uma função
de mérito, que quando analisamos a penalidade dela vemos que também tem uma forma semelhante
à Lagrangiana. Dessa forma temos a habilidade de usar essas funções de maneira similar ao uso
nos métodos empregados no LANCELOT (\ref{sec_lancelot}) e DONLP2 (\ref{sec_dosnlp2}).