% RESUMO--------------------------------------------------------------------------------

\begin{resumo}[\larger Resumo]
  % \begin{SingleSpacing}

  %   Não altere esta seção do texto--------------------------------------------------------
  %   \imprimirautorcitacao. \imprimirtitulo. \imprimirdata. \pageref {LastPage} f. \imprimirprojeto\ – \imprimirprograma, \imprimirinstituicao. \imprimirlocal, \imprimirdata.\\
  %   ---------------------------------------------------------------------------------------


  O método SCP pode ser posto em uma condição histórica renascentista, uma vez que ele se propõe a usar métodos lineares que há muito foram deixados de lado. Mesmo usando algo que por vezes foi considerado uma tecnologia defasada, o método foi capaz de atingir bons resultados. Esse trabalho tem como foco o estudo dos fundamentos, funcionamento e implicações do método em um contexto contemporâneo. Um produto desse estudo foi a confecção de um código que tenta equivaler-se ao método proposto, que data de 2010. Tanto o estudo quanto a confecção do código resultaram em um ferramental teórico e prático a respeito não só do SCP, mas também à outros métodos, como o do gradiente conjugado e o método sequencial quadrático. Análises sobre o funcionamento do método e a conjuntura das ferramentas usadas em sua construção foram feitas a fim de compreender não só as possibilidades advindas do método, mas também seu percusso e o que ele significa na história da otimização. 



  \vspace{0.7cm}
  \textbf{Palavras-chave}: Otimização contínua. Problemas não lineares. Programação.

  % \end{SingleSpacing}
\end{resumo}
