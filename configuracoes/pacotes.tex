% REFERÊNCIAS------------------------------------------------------------------
\usepackage[%
    alf,
    abnt-emphasize=bf,
    bibjustif,
    recuo=0cm,
    abnt-url-package=url,       % Utiliza o pacote url
    abnt-refinfo=yes,           % Utiliza o estilo bibliográfico abnt-refinfo
    abnt-etal-cite=3,
    abnt-etal-list=3,
    abnt-thesis-year=final
]{abntex2cite}                  % Configura as citações bibliográficas conforme a norma ABNT
% PACOTES----------------------------------------------------------------------

\usepackage{booktabs}                                       % Réguas horizontais em tabelas
\usepackage{color, colortbl}                                % Controle das cores
\usepackage{float}                                          % Necessário para tabelas/figuras em ambiente multi-colunas
\usepackage{graphicx}                                       % Inclusão de gráficos e figuras
\usepackage{icomma}                                         % Uso de vírgulas em expressões matemáticas
\usepackage{indentfirst}                                    % Indenta o primeiro parágrafo de cada seção
\usepackage{microtype}                                      % Melhora a justificação do documento
\usepackage{multirow, array}                                % Permite tabelas com múltiplas linhas e colunas
\usepackage{subeqnarray}                                    % Permite subnumeração de equações
\usepackage{lastpage}                                       % Para encontrar última página do documento
\usepackage{verbatim}                                       % Permite apresentar texto tal como escrito no documento, ainda que sejam comandos Latex

\usepackage{amsfonts, amssymb, amsmath}                     % Fontes e símbolos matemáticos
\usepackage{amsthm}
\usepackage{amscd,latexsym}

%\usepackage{geometry} % just for not getting an overfull box
%\usepackage{bm}

    
    
\usepackage[algoruled, portuguese]{algorithm2e}             % Permite escrever algoritmos em português
\usepackage[bottom]{footmisc}                               % Mantém as notas de rodapé sempre na mesma posição
\usepackage{ae, aecompl}                                    % Fontes de alta qualidade
\usepackage{latexsym}                                       % Símbolos matemáticos
\usepackage{lscape}                                         % Permite páginas em modo "paisagem"
%\usepackage{picinpar}                                      % Dispor imagens em parágrafos
%\usepackage{scalefnt}                                      % Permite redimensionar tamanho da fonte
%\usepackage{subfig}                                        % Posicionamento de figuras
%\usepackage{upgreek}                                       % Fonte letras gregas
\usepackage[font={footnotesize,bf},center,singlelinecheck=false,tableposition=bottom]{caption}
\usepackage{tabularx}

\usepackage[utf8]{inputenc}                                 % Codificação do documento
\usepackage{fontspec}
%\usepackage[T1]{fontenc}
%\usepackage[scaled]{helvet}                                % Usa a fonte Helvetica
%\usepackage{times}                                         % Usa a fonte Times
%\usepackage{palatino}                                      % Usa a fonte Palatino
%\usepackage{lmodern}                                       % Usa a fonte Latin Modern
%\usepackage{accanthis}
\setmainfont{Times New Roman}[SizeFeatures={Size=12}]
    
\newfontface \helveft{helvetica-bold.otf}
\renewcommand{\captionfont}{\fontsize{8.5}{6}\selectfont{}\helveft}
\renewcommand{\captionlabelfont}{\fontsize{8.5}{6}\selectfont{}\helveft}
    
%\usepackage{indentfirst}

    
%\renewcommand{\familydefault}{\sfdefault}
%\setmainfont{Accanthis ADF Std No3}
%\setromanfont{Accanthis ADF Std No3}
%\setsansfont{Accanthis ADF Std No3}
%\setmonofont{Accanthis ADF Std No3}
    
%% Times New Roman
%\setromanfont[
%BoldFont=timesbd.ttf,
%ItalicFont=timesi.ttf,
%BoldItalicFont=timesbi.ttf,
%]{times.ttf}
%% Arial
%\setsansfont[
%BoldFont=arialbd.ttf,
%ItalicFont=ariali.ttf,
%BoldItalicFont=arialbi.ttf
%]{arial.ttf}
%% Courier New
%\setmonofont[Scale=0.90,
%BoldFont=courbd.ttf,
%ItalicFont=couri.ttf,
%BoldItalicFont=courbi.ttf,
%Color={0019D4}
%]{cour.ttf}



\newtheorem{theorem}{Teorema}
%\newtheorem{acknowledgement}[theorem]{Acknowledgement}
%\newtheorem{algorithm}[theorem]{Afirmação}
%\newtheorem{axiom}{Axioma}
%\newtheorem{case}[theorem]{Caso}
%\newtheorem{claim}[theorem]{Claim}
%\newtheorem{conclusion}[theorem]{Conclusion}
%\newtheorem{condition}[theorem]{Condition}
%\newtheorem{conjecture}[theorem]{Conjectura}
%\newtheorem{corollary}{Corolário}
%\newtheorem{criterion}[theorem]{Criterion}
\newtheorem{definition}{Definição}
%\newtheorem{example}{Exemplo}
%\newtheorem{exercise}{Exercício}
%\newtheorem{lemma}{Lema}
%\newtheorem{notation}{Notação}
%\newtheorem{problem}[theorem]{Problem}
%\newtheorem{proposition}{Proposição}
%\newtheorem{remark}{Observação}
%\newtheorem{solution}[theorem]{Solução}
%\newtheorem{summary}[theorem]{Summary}
%\newenvironment{proof}[1][\textbf{Demonstração:}\hspace{0.2cm}]{\textit{#1}}{\begin{flushright}\itshape{QED.}\end{flushright}}

\usepackage[short,nocomma]{optidef}
\usepackage{blindtext}
\usepackage{pgfplots}
\pgfplotsset{compat=1.9}

 
%\usepackage{xparse}
% load showframe just to show where the box borders are
%\usepackage{showframe}

%\newsavebox{\fminipagebox}
%\NewDocumentEnvironment{fminipage}{m O{\fboxsep}}
% {\par\kern#2\noindent\begin{lrbox}{\fminipagebox}
%  \begin{minipage}{\dimexpr#1-2\fboxsep-2\fboxrule}\ignorespaces}
% {\end{minipage}\end{lrbox}%
%  \fbox{\usebox{\fminipagebox}}%
%  \par\kern#2 }