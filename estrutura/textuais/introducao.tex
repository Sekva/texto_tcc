% INTRODUÇÃO-------------------------------------------------------------------

\chapter{\larger Introdução}
\label{chap:introducao}
		
Os processos industriais potencializados no início do século XX guiado pelos frontes das duas grandes gerras trouxe um solavanco ao desejo de solução de problemas de caráter otimizável. Dois personagens em cenários distintos se sobressaem dando partida a corrida armamentista dos métodos de otimização.

Por outro lado, de um ponto de vista teórico os métodos de otimização nascem junto com o Cálculo Diferencial advindo do Principia (Newton) e reformulado por seus colaboradores, Euler, Cauchy, Weierstrass, Lagrange, entre outros. O mecanismo em si ainda precisava de um interlocutor, o computador, a máquina que decifraria os enigmas da fenomenologia. Com a evolução dos sistemas processuais booleanos advieram a cepa de uma nova era, o modernismo pragmático das máquinas digitais possibilitaram um novo horizonte a ser concebido pelas mentes mais audazes e, convenientemente, mais bem preparadas de toda a história humana, os cientistas do meio do século XX. 

Apenas depois de quase 30 anos de avanços teóricos e práticos é que pudemos contar com métodos que se equivaleriam ao clássico método de Newton no que se refere a solução de problema de busca linear. O efeito desse avanço nos colocou em um movimento que só fez acelerar. Desde o combalido método Simplex, enviesando entre os métodos de Newton e seus correlacionados, estamos em uma busca contínua por um modo de encontrar ótimos de funções de formas cada vez mais eficientes. Por essa busca e pelo sucesso dela é que temos hoje computadores funcionais em dispositivos móveis e uma janela de aplicabilidades dos modelos de otimização evoluindo dia após dia. 

É nesse cenário que viemos apresentar um método que foi gerado para ser aplicado a um espectro grande de modelos pela amplitude de seus parâmetros. Visando uma aplicação à problemas não lineares e em que as funções que representam o modelo não sejam necessariamente conexas, o SCP se propõe como uma ferramenta robusta e metamorfa. Pois o algoritmo no qual está erguido faz uso de um modo sofisticado dos elementos que compõem todos os métodos clássicos, e mais, ele se supõe ser ainda mais eficaz que cada um deles.

Entendo que só se poderia apresentar os aspectos técnicos do SCP mediante uma abreviada introdução ao teor histórico que fomentou as ideias por trás do ideal do algoritmo e, que também, seria necessário apresentar estruturas fundamentais à otimização matemática, em seu próprio território, dentre outras estruturas que se aliciam à técnica, seja pelo suporte ao mecanismo teórico, seja pela praticidade de linguagem. Apresentamos no Capítulo \ref{chap:fundamentacaoTeorica} um relato histórico e teórico que sustenta o SCP em seus alicerces mais amplos. Iniciando com a ideia prática de resolver um problema e embarcando numa discussão indo até o problema de fazer estimativas de um ente matemático clássico que se revela indomesticável quase, a Hessiana.

Seguindo uma estrutura textual que possibilite a compreensão do que foi produzido ao longo de seis meses de dedicação erguemos o método científico que utilizamos no Capítulo \ref{chap:metodologia} e, em seguida, no Capítulo \ref{chap:analise_discussiva} apresentamos nosso objeto central, a análise analítica do método mediante uma sentença prática. Fazendo uso de lógica de programação foi estruturado um algoritmo auxiliar que serviu para imergir aos instrumentos intrínsecos do SCP e revelar seus mecanismos e manobras. Disso seguimos para uma modelagem de um método construído com um algoritmo inédito, que traduz o SCP de uma forma que possibilite a sua codificação em uma linguagem que consiga compilar suas rotinas.

Além de se aproximar com muita eficiência de um algoritmo que revele o SCP numa plataforma pública e de fácil acesso, os frutos desse trabalho transbordam na área teórica, no que diz respeito a formação de recursos humanos, e no requisito prático, no que diz respeito a elaboração de um código que lê o SCP via lógica de programação. 

Ademais, desejamos que esse trabalho possa vir a servir de inspiração a outros trabalhos que evoluam ainda mais os avanços que atingimos e que possam trazer uma estrutura mais sólida para a área da otimização contínua.