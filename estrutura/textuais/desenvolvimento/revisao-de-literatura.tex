% REVISÃO DE LITERATURA--------------------------------------------------------

\chapter{\larger Revisão Literária}
\label{chap:fundamentacaoTeorica}

É uma boa prática iniciar cada novo capítulo com um breve texto introdutório (tipicamente, dois ou três parágrafos) que deve deixar claro o quê será discutido no capítulo, bem como a organização do capítulo.
Também servirá ao propósito de "amarrar"{} o conteúdo deste capítulo com o conteúdo do capítulo imediatamente anterior.

\section{A programação sob o viés matemático}
Esboço de um resumo dos dois primeiros capítulos do livro de otimização, o que vai justificar muita coisa na verdade temos que ver depois v
\subsection{Otimizando à uma variável}
\subsection{Otimizando à mais de uma variável}

%subscheqchion

\section{As primeiras três décadas da programação}

Na década de 1930 o mundo encontrava-se em um estado onde problemas de gerenciamento
se tornaram muito complexos, e tão pouco estruturados, que modelos matemáticos e o poder
computacional existente na época foram postos à prova. A partir daí o campo de
pesquisa responsável por esses assuntos começou a tomar forma e ganhar mais importância.
Dentre os vários nomes dados a essa área de pesquisa, podemos encontrar pesquisa operacional,
análise de sistemas, análise de custo-benefício e análise de custo-efetividade. Durante essa
década, onde inciava-se a segunda guerra mundial, problemas como a alocação de recursos
limitados entre atividades simultâneas sob certas restrições eram comuns.
TODO: prólogo de preparação das subseções


\subsection{Kantorovich}

Em 1939, problemas de otimização de produções com base nas reservas da indústria (materiais,
trabalhadores e equipamentos) se tornaram essenciais no sustento do estado Soviético.
Segundo Kantorovich \cite{kantorovich1939}, existem duas formas de melhorar a eficiência
de uma fábrica. A primeira é melhorando a tecnologia empregada nos meios de produção, e a
outra é melhorando a organização do planejamento e da produção.

Antes disso, Kantorovich havia percebido que uma grande quantidade de problemas, de diversas
áreas, levavam à formulação de problemas de extremos. Mas ele também notou que esses problemas,
embora similares à problemas da análise matemática, quando se tentava resolver
usando os mesmos processos matemáticos, formalmente definidos, se tornavam inúteis, pois era
deveras custoso no aspecto computacional. Dentre um dos exemplos está a resolução de milhares
ou milhões de sistemas de equações.

Ele foi capaz de reduzir vários dos problemas que encontrou em três tipos.
Partindo do problema de alocação de tempo de uso de máquinas para a produção de partes que compõem
um dado produto, foi formulado um problema genérico considerando um número
\(n\) de máquinas, e a produção de \(m\) partes diferentes para confecção do produto. Usando
a \(i\)-ésima máquina para produzir apenas a \(k\)-ésima parte durante um dia inteiro de trabalho,
seria produzido \(\alpha_{i, k}\) partes. O problema consiste em dividir o tempo de trabalhado de cada
máquina de forma que possa ser construído o máximo possível de produtos.


Considerando \(h_{i, k}\) como a fração do dia de trabalho que a máquina \(i\) passou produzindo
a parte \(k\) sendo a variável que desejamos otimizar. Por motivos óbvios, ela deve ser positiva.
E também, por não podermos considerar que uma máquina esteja parada durante o dia de trabalho, temos
duas restrições:

\begin{enumerate}
  \item \(h_{i, k} \geq 0\), \(i =1, ..., m \);
  \item \(\sum_{k=1}^m h_{i, k} = 1\), \(i =1, ..., m \);
  \label{restricoes_a}
\end{enumerate}

Observando que a restrição \(2.\) significa que a soma das frações \(h_{i, k}\) deve resultar
em 1, visto que a unidade de tempo é o dia. Portanto, temos que a quantidade da \(k\)-ésima parte
produzida no dia é:

\begin{equation}
  z_k = \sum_{i=1}^n \alpha_{i, k} h_{i, k}.
  \label{zk}
\end{equation}


\par Para que se obtenha o máximo de produtos completos, é necessário que se tenha a mesma quantidade de
cada tipo de parte, do contrário foi perdido tempo produzindo partes extras. Com isso temos o
\textit{Problema A}, cuja configuração é dada pelas restrições a seguir:

\begin{center}
  Encontrar \(h_{i, k}\) tal que as partes \(z_k\) sejam todas iguais, isto é \(z_{k_1} = z_{k_2} \)
  para \(k_1, k_2 \in {1, ..., m}\). Onde \(z_k\) é dado pela expressão \ref{zk} e adicionalmente
  \(h_{i, k}\) satisfazendo as restrições descritas em \ref{restricoes_a}.
\end{center}

Problemas como o uso racional de combustível e distribuição ótima de terras aráveis
também foram usados como exemplos de problemas que podem ser reduzidos a mesma forma
que o Problema A.

Caso tenhamos mais condições limitantes, temos um outro modelo esperado por Kantorovich.
Considerando o mesmo problema de produção anteriormente definido, e agora com limite
máximo de energia gasta \(C\) para todas as máquinas e, \(c_{i, k}\) sendo o consumo de energia
da máquina \(i\) produzindo a parte \(k\) por um dia inteiro de trabalho, temos que o
Problema B é o Problema A com mais uma restrição:

\begin{align}
  \sum_{i=1}^n \sum_{k=1}^m c_{i, k} h_{i, k} \leq C
\end{align}

Podem existir vários significados sobre as quantidades em torno de \(C\), como por exemplo
a quantidade de pessoas disponíveis para trabalho, consumo de água ou qualquer recurso
necessário para a produção. Da mesma forma podemos ter diversas restrições como esta,
o que ainda é classificado como um problema tipo Problema B.

Ainda há um terceiro modelo de problema, quando uma mesma máquina pode produzir ao mesmo tempo
mais de uma parte, e que ela pode ser configurada para operar de diferentes métodos, com cada
método resultando em números diferentes de partes. Considerando \( \gamma_{i, k, l} \) o número de
\(k-esimas\) partes, produzidas na \(i-esima\) máquina, utilizando a \(l-esima\) configuração,
é preciso encontrar a fração de horas do dia trabalhado \(h_{i, l}\) que uma máquina \(i\).
O números de \(k-esimas\) partes é dado por \( z_k = \Sigma_{i, l} \gamma_{i, k, l} h_{i, l}\). São aplicados as
restrições da mesma forma que para o Problema A e o Problema B. Com isso temos o Problema C:

\begin{align}
  max(z_k); \\
  z_1 = z_2 = ... = z_m; \\
  z_k = \sum_{i, l} \gamma_{i, k, l} h_{i, l} \\
  h_{i, l} \geq 0; \\
  \sum_l h_{i, l} = 1; \\
\end{align}


Para a resolução de tais problemas, Kantorovich desenvolveu um método simples. A partir do
Problema A, é certo que existem um conjunto de multiplicadores \( \lambda_1, \lambda_2, ..., \lambda_m \) com
cada um correspondendo a uma parte produzida, que são capazes de levar à solução do problema.
A partir deles podemos observar cada máquina \(i\) e escolher os multiplicadores que geram os
maximos entre \( \lambda_1\alpha_{i, 1}, \lambda_2\alpha_{i, 2}, ..., \lambda_m\alpha_{i, m} \). Em seguinda, os multiplicadores \( \lambda_k \)
observados nas máquinas \(i\) que resultaram em maximos são postos como \(h_{i, k}\). Feito isso,
fica muito mais fácil determinar os outros valores para \( h_{i, k} \).

TODO: falar mais sobre esse método? ainda não entendi direito


\subsection{Dantzig}

\section{Programação não linear - O surgimento}
\section{Os pilares arquitetônicos do SCP}
\section{O modernismo sob o ponto de vista da programação (Uma ideia do o quê que a gente tem de moderno)}
