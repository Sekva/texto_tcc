% METODOLOGIA------------------------------------------------------------------

\chapter{\larger Percurso Metodológico}
\label{chap:metodologia}
Cada capítulo deve conter uma pequena introdução (tipicamente, um ou dois parágrafos) que deve deixar claro o objetivo e o que será discutido no capítulo, bem como a organização do capítulo.

\begin{itemize}
  \item o método analitico clássico, conjecturas/hiposteses -> teses
  \item as bases teoricas, como cheguei ao scp
  \item as bases teoricas, como refinei a compreensao do scp usando do meu entendimento programando
  \item a construção do código:
  \item escolha da linguagem
  \item como construi o algoritmos
  \item quais parametros usados para testar o código
\end{itemize}

\section{DELINEAMENTO DA PESQUISA}
\label{sec:titSecDelPesq}

\begin{itemize}
  \item definição do projeto
  \item tema do projeto
  \item o objetivo do projeto
  \item a viabilidade do projeto
  \item o cronograma do projeto
  \item a margem de erro do projeto
\end{itemize}

\section{COLETA E TRATAMENTO DE DADOS}
\label{sec:titSecColDad}

\begin{itemize}
  \item como encontrei os textos
  \item que tipo de leitura foi feita dos textos
  \item elaboração da análise qualitativa e quantitativa dos textos
  \item conclusão sobre a eficiencia dos métodos usados pra pesquisa
\end{itemize}