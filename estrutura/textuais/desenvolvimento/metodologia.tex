\chapter{Percurso Metodológico}
\label{chap:metodologia}

%\begin{itemize}
%  \item o método analítico clássico, conjecturas/hipóteses -> teses
%  \item as bases teóricas, como cheguei ao SCP
%  \item as bases teóricas, como refinei a compreensão do SCP usando do meu entendimento programando
%  \item a construção do código:
%  \item escolha da linguagem
%  \item como construí o algoritmos
%  \item quais parâmetros usados para testar o código
%\end{itemize}
\noindent
Para a produção desse trabalho foi considerado o método analítico, inaugurado por Descartes. O método, de forma geral, considera a criação de conjecturas acerca do tema, e por meio de passos bem elaborados, chega à construção de uma tese.

A introdução ao tema foi feita no ano de 2019, junto ao início de um projeto de pesquisa realizado por um grupo de alunos do curso de Bacharelado da Ciência da Computação da UFAPE. No início do projeto foi feita uma exposição inicial sobre o que deveria ser feito, quais temas buscar e onde buscar. Após essa orientação inicial, eram feitas reuniões semanais ondem eram apresentados os resultados de uma pesquisa superficial sobre as aplicações de métodos de otimização em situações práticas, como em controladores de máquinas de precisão e otimização de classificadores. Essas pesquisas semanais foram feitas a partir da leitura de artigos encontrados em buscas em bases científicas, as quais são acessíveis por intermédio do portal de periódicos da CAPES. Depois de feita uma quantidade que foi considerada suficiente dessas reuniões, cada aluno do grupo de pesquisa foi incumbido de fazer a escolha de um método dos que foram apresentados durante as reuniões. O método escolhido foi esse que resultou na produção desse trabalho.

Terminado o projeto de pesquisa e já escolhido o método, os esforços foram voltados para a confecção desse trabalho, iniciando uma nova fase de estudos sobre o SCP. Inicialmente foi feita uma leitura um pouco mais aprofundada, buscando conhecer, de forma simples, os fundamentos usados pelos autores para a construção do método, bem como compreender os problemas para os quais o método foi aplicado, a fim de ter o entendimento do meio em que ele foi concebido e também a identificação de métodos concorrentes em relação à aplicação, e métodos semelhantes em relação às ferramentas utilizadas para seu funcionamento. Em paralelo a isso, enquanto se buscava identificar os fundamentos do SCP foi sendo construído um conhecimento básico sobre o funcionamento do método, culminando em um seminário sobre o mesmo.

Quando houve a decisão de que o tema escolhido para este trabalho seria o estudo do SCP, foi feito um delineamento do que seria feito para a pesquisa. Além disso, também houve a decisão da implementação de um código para o algoritmo. Um dos motivos para essa implementação, além da criação de uma versão pública e acessível do algoritmo, é auxiliar no entendimento do funcionamento do método e de suas bases, uma vez que experimentos empíricos trazidos por essa implementação ajudam na compreensão e indicação de comportamentos do método.

Para a implementação do algoritmo foi escolhida uma linguagem de programação que cumpria alguns requisitos para que se pudesse ter ao final o que se esperava do código. Dentre os requisitos, temos que a linguagem deve ser capaz de produzir uma biblioteca dinâmica para que pudesse ser usada por qualquer outra linguagem com capacidade de utilizar uma biblioteca dinâmica, que a linguagem deve permitir o porte do código do algoritmo para qualquer outra plataforma suportada pelo compilador, que seja uma linguagem auto-explicativa no melhor que ela puder ser e que tenha recursos que facilitem a seu uso e a fácil produção de código de máquina seguro e eficiente. A linguagem escolhida para ser usada foi Rust, uma linguagem moderna, mas voltada para a construção de sistemas críticos, como bibliotecas para programas, sistemas operacionais e aplicações em embarcados. A linguagem cumpre todos os requisitos já citados, além de que já se existia um conhecimento prévio dela.

A construção do código foi feita em paralelo ao estudo das ferramentas usadas para elaborar o método. Uma fez terminado o estudo de uma das bases elencadas pelos autores do SCP, era implementado o código equivalente ao que foi estudado, apelando quando necessário para um estudo mais aprofundado. Desta forma, existia um vínculo entre o artigo onde foi apresentado o SCP, o estudo feito a respeito do que se foi empregado no artigo e a construção do código. Enquanto construído, o código passava por testes, os quais eram feitos para verificar a corretude de cada parte. Para cada parte do código, eram usados problemas específicos para que fosse testado, tentando ao máximo garantir o funcionamento correto de cada parte.

\section{Delineamento da Pesquisa}
\label{sec:titSecDelPesq}

%\begin{itemize}
%  \item definição do projeto
%  \item tema do projeto
%  \item o objetivo do projeto
%  \item a viabilidade do projeto
%  \item o cronograma do projeto
%  \item a margem de erro do projeto
%\end{itemize}


\noindent
A definição do escopo do projeto desse trabalho foi feita ainda durante o projeto de pesquisa, bem como o tema a ser estudado. A escolha foi feita a partir do interesse criado sobre o método e sobre o que levou os autores a elaborarem o mesmo. Dada a complexidade do método é natural que se levantem discussões sobre o funcionamento e o que levou a sua construção, e por isso um interesse maior.

O objetivo desse trabalho é, em primeiro lugar, a compreensão do método SCP, seus fundamentos, utilidade e implicações de sua existência. Em segundo lugar, é ter como um código acessível, público e útil para os problemas que o SCP se propõe em resolver. Ao considerarmos que o método é de fato competitivo em relação aos outros métodos que temos hoje em dia, um código público e acessível do mesmo é de grande importância para a evolução da comunidade científica no que diz respeito aos estudos de métodos de otimização, uma vez que, por exemplo, testes numéricos podem ser realizados entre a implementação do SCP e outro método que venha a existir ou que já exista, produzindo um melhor entendimento a eficiência e capacidade de aplicação dos métodos. O projeto se torna viável quando é percebido que as ferramentas utilizadas para a construção do método, no que diz respeito às suas implementações, não são complexas a ponto de gerar algum impedimento. O que poderia gerar uma possível inviabilidade para a execução do projeto seria a tentativa de compreensão total dessas ferramentas. Mas ao considerar que o estudo dessas ferramentas é necessário até um certo ponto, e não em sua completude, descartamos essa inviabilidade.

O cronograma considerado para a execução desse trabalho teve inicio após o término do projeto de pesquisa já mencionado. Inciado em Janeiro de 2021, o primeiro passo do cronograma foi fazer uma releitura no artigo onde foi descrito o SCP, tentando listar o máximo de pontos citados para um estudo posterior. Essa releitura ditou a ordem em que seriam feitas as pesquisas sobre os pontos elencados, uma vez que essa foi a escolha dos autores para a apresentação do método. Feita essa releitura, inciou-se apenas os estudos dos primeiros pontos inicialmente elencados, resultando em apresentações semanais sobre o que foi estudado. Ao chegar ao ponto em que o assunto coincidia com o primeiro passo a ser tomado para a construção do código, foi de fato iniciado a construção, a fim de manter um paralelismo entre o código que estava sendo construído e o assunto que estava sendo estudado, garantindo uma melhor extração de conhecimento por ambos os meios. E assim foi feito até o término dos estudos e a finalização do código em uma primeira versão.

Como o objetivo primário desse trabalho é gerar conhecimento sobre um método moderno e tudo o que isso representa, é possível considerar metas não alcançadas. Dentre elas, a garantia que ao final do projeto se tenha o código proposto completamente funcional, ou ainda, a contemplação de todos os assuntos elencados durante a releitura do artigo do SCP em igual profundidade, ou deixando algum de fora quando se viu algum outro com maior prioridade.

\section{Coleta e Tratamento de Dados}
\label{sec:titSecColDad}

%\begin{itemize}
%  \item como encontrei os textos
%  \item que tipo de leitura foi feita dos textos
%  \item elaboração da análise qualitativa e quantitativa dos textos
%  \item conclusão sobre a eficiência dos métodos usados pra pesquisa
%\end{itemize}

\noindent
A busca pelo material de estudo, como já dito, iniciou-se por uma releitura do artigo, onde foram extraídas referências pertinentes à outros artigos e livros. Com esse material em mãos, foi sendo feita uma busca em profundidade nas referências de cada tema estudado, afim de se chegar a uma origem concreta sobre o que se era estudado, bem como o percusso histórico. Todos os textos foram acessados por meio do portal de periódicos da CAPES.

Para cada texto visto, foi feita uma pequena avaliação qualitativa e quantitativa acerca do que se era expressado, bem como seu valor para a pesquisa que estava sendo feita. Essa avaliação garantiu a existência de uma base de conhecimento sólida para a execução do projeto. Dessa forma, podemos concluir que mesmo que seja um processo custoso, o método aplicado é robusto e gera um resultado que pode ir além das expectativas vistas em seu planejamento, o que por si só pode ser visto como um indicativo de uma garantia de sucesso em algum aspecto do trabalho.