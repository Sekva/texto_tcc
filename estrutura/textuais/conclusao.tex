% CONCLUSÃO--------------------------------------------------------------------

\chapter{Conclusão}
\label{chap:conclusao}

\noindent
Como efeito do estudo epistemológico da literatura acerca do SCP partindo
do que se sugere ser a origem, que podemos considerar com ancestralidade à
Newton, os primeiros movimentos pós-guerra que introduzem a programação linear
e suas metalinguagens. Temos uma análise qualitativa arquitetada sobre pilares
de estrutura lógico-computacional, o que permite dissertar e descrever o
algoritmo que é apresentado no trabalho \cite{Still2010}. Também nos permitiu,
uma vez acessado o ideal do método, construir em uma configuração própria, o
que configura o significado de inovação, um código que traduz o algoritmo.
E podemos também conjecturar que há um quê de inovador, em alguma dimensão,
do algoritmo primal quando construído o código que sugerimos.

Num outro edifício fundamentado na funcionalidade e eficácia do método temos
uma análise quantitativa que permeia as escalas de profundidade que pudemos
alcançar com as ferramentas que tínhamos à disposição, a saber, a proposta de
código ainda não finalizada. 

\section{Trabalhos Futuros}
\label{sec:trabalhosFuturos}

\noindent
Tendo em vista o amplo potencial destacado do SCP no capítulo
\ref{chap:analise_discussiva}, sabemos que problemas em que a função
objetivo seja convexa estão todos dentro do escopo de trabalhos futuros. Além
do mais, pelo que conferimos, desde que a função objetivo seja continuamente
diferenciável, e ademais, com restrições dadas por funções suaves,
problemas de modelagem matemática com expressões ao menos algébricas podem
ser atingidos pelo SCP. Dentre esses problemas temos o problema de N-Corpos.
Especificamente o problema de encontrar configurações centrais para N
suficientemente grande, com o intuito de determinar sobre a finitude das
classes de configurações centrais abordando o fator simétrico.

No que se refere à questão computacional, é pretendido não só finalizar o
código que propomos, como também expandir a abordagem das estimativas para a
matriz hessiana. O que pode vir a simplificar o algoritmo, e como consequência
quiçá amplificar, ainda mais, o leque de aplicações de problemas solúveis pelo
SCP.

\section{Considerações Finais}
\label{sec:consideracoesFinais}

\noindent
Uma plausível convergência das concatenações do que foi exposto nos capítulos
\ref{chap:fundamentacaoTeorica} e \ref{chap:analise_discussiva}, pode ser uma
alusão à imersão das contendas tanto teóricas quanto práticas de uma área de
pesquisa matemática-computacional que é a otimização. Nesses termos qualquer
avanço que se possa, teórico ou prático, nessa área aponta rapidamente para
uma evolução em todas as áreas que hoje fazem uso da otimização como
sub-rotina de sua metodologia, o que confere aos avanços que aqui apresentamos
como sugestivos a serem aplicados em curto prazo em áreas díspares, visto, por
exemplo, que modelos de crescimento populacional atingem da sociologia à
biomedicina com equivalente potencial de valor. E os modelos mais modernos
possuem intrinsecamente como função objetivo, uma função que compete à
programação não linear. Isso nos mostra que cedo ou tarde os métodos de
otimização terão que ocupar o espaço de técnicas viáveis para resolver
problemas de fenômenos contumaz modernos.

Como uma conclusão óbvia, o trabalho que foi realizado tanto é moderno como
necessário. Ultrapassando o viés primário de evidenciar as competências de um
discente em sua conclusão de graduação e aflorando para um desfiladeiro de
novos horizontes e uma certeza, de que dentre todas as possibilidades, esse se
apresenta em tempo como o trabalho que me preparei para realizar ao longo de
minha formação acadêmica.