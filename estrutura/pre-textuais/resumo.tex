% RESUMO--------------------------------------------------------------------------------

\begin{resumo}[\larger Resumo]
%\begin{SingleSpacing}

% Não altere esta seção do texto--------------------------------------------------------
%\imprimirautorcitacao. \imprimirtitulo. \imprimirdata. \pageref {LastPage} f. \imprimirprojeto\ – \imprimirprograma, \imprimirinstituicao. \imprimirlocal, \imprimirdata.\\
%---------------------------------------------------------------------------------------

O Resumo é um elemento obrigatório em tese, dissertação, monografia e TCC, constituído de uma seqüência de frases concisas e objetivas, fornecendo uma visão rápida e clara do conteúdo do estudo. O texto deverá conter no máximo 500 palavras e ser antecedido
pela referência do estudo. Também, não deve conter citações. O resumo deve ser redigido em parágrafo único, espaçamento simples e seguido das palavras representativas do conteúdo do estudo, isto é, palavras-chave, em número de três a cinco, separadas entre si por ponto e finalizadas também por ponto. Usar o verbo na terceira pessoa do singular, com linguagem impessoal, bem como fazer uso, preferencialmente, da voz ativa. Texto contendo um único parágrafo.\\

\textbf{Palavras-chave}: Palavra. Segunda Palavra. Outra palavra.

%\end{SingleSpacing}
\end{resumo}

% OBSERVAÇÕES---------------------------------------------------------------------------
% Altere o texto inserindo o Resumo do seu trabalho.
% Escolha de 3 a 5 palavras ou termos que descrevam bem o seu trabalho
%
% Adequação das palavras chaves, a biblioteca solicitou que todos adequem as
% palavras chaves de acordo com as cadastras no sistema de disponibilização de
% trabalhos da biblioteca.

% Segue as instruções: A busca por palavra chave é feita da seguinte forma:
% Acessar o site da biblioteca nacional: http://acervo.bn.br/sophia_web/; Clicar
% em autoridades; Em busca por autoridade selecionar "termo tópico"; Selecionar
% "iniciar com"; Digitar a palavra que se quer verificar se está cadastrada (ex:
% sistemas) e após buscar; Note que no ex. "sistemas" apareceram várias opções
% (Sistemas auto-organizáveis, Sistemas CAD/CAM, Sistemas de avaliação de risco
% de crédito), o número 16 corresponde à "sistemas de computação". Clicando nesse
% termo aparecem 3 abas, as quais se pode verificar se realmente é o termo
% procurado; Na aba MARC Tag usamos o número 150 para a palavra-chave em
% português e o número 750 para a palavra-chave em inglês.
