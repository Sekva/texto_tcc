% ABSTRACT--------------------------------------------------------------------------------

\begin{resumo}[\larger Abstract]

  The SCP method can be placed in a renaissance historical condition, since it purports to use linear methods that have long been discarded. Even using something that was sometimes considered outdated technology, the method was able to achieve good results. This work focuses on the study of the fundamentals, functioning and implications of the method in a contemporary context. The product of this study was the making of a code that tries to match the proposed method, which dates from 2010. Both the study and the making of the code resulted in theoretical and practical tools regarding not only the SCP, but also other methods, such as conjugate as the gradient and the sequential quadratic method. Analyzes of the method's functioning and the conjunction of the tools used in its construction were carried out in order to understand not only the possibilities arising from the method, but also its path and what it means in the history of optimization. 

  \vspace{0.7cm}
  \textbf{Keywords}: Continuous optimization. Nonlinear Problems. Programming.

\end{resumo}

