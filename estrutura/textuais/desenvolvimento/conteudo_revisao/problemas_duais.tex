\section{Problemas lineares duais}
\label{sec_prob_dual}
Qualquer problema linear possui um outro problema atrelado a ele, o problema dual \cite{fischetti2019}.
Considerando a ideia de que queremos minimizar uma quantidade na reta real e, por
simplicidade, assumimos que este mínimo existe. Podemos dizer então que este valor
mínimo também é um valor máximo para todas as quantidades menores que aquele mínimo. Com
isso, se tivermos a capacidade de gerar um problema de maximização onde o máximo desse
problema seja o mínimo do que queremos minimizar inicialmente, e se o que for importante de
fato seja o valor, estamos feitos. A efeito de exemplo, podemos usar um exemplo clássico de
um fazendeiro e o comprador de seus produtos, onde o primeiro quer maximizar seus lucros, e
o segundo quer minimizar seus custos. É certo que ambas as partes querem fechar um acordo, e,
para isso, o mínimo do comprador deve ser o máximo do fazendeiro.

Considerando o seguinte problema linear de minimização, que chamaremos de problema primal:

\vspace{-15pt}
\begin{mini!}
{x}{ c^Tx \label{duall_obj}}{\label{prob_duall}}{}
\addConstraint{x}{\in \mathbb{R}^n}{\label{r1_duall}}
\addConstraint{x}{\in P := \{Ax = b,\ x \geq \overrightarrow{\mathrm 0}\}}{\label{r2_duall}}
\addConstraint{b}{\in \mathbb{R}^m}{\label{r3_duall}}
\addConstraint{A}{\in \mathbb{R}^{m \times n}}{\label{r4_duall}}
\end{mini!}

Podemos escrever o problema de achar um máximo de outro problema, que chamaremos de dual, como:

\begin{equation}
\underset{x}{\mathrm min}\{c^Tx\ |\ x \in P \} = \underset{c_0}{\mathrm max}\{c_0\ |\ c^Tx \geq c_0\ \forall x \in P\}
\end{equation}

Com isso, já podemos perceber alguns fatos sobre ambos os problemas. Se considerarmos que o
problema primal é irrestrito, ou seja, se \(\underset{x}{\mathrm min}\{c^Tx\ |\ x \in P\} = - \infty \),
então não seremos capazes de encontrar um \(c_0\) que satisfaça. Portanto, se o problema primal
for irrestrito, então o problema dual é inalcançável. O caminho oposto também é verdade, se
considerarmos o problema dual irrestrito, então \(c_0 = \infty\), logo não existe um mínimo para o problema
primal.

Como queremos encontrar \(c_0\) de forma que \(c^Tx \geq c_0\), a partir dos sistemas de equações
formados pelas restrições podemos construir expressões limitantes para \(c^Tx\). Novamente,
considerando o mesmo problema (\ref{prob_duall}), podemos multiplicar todo o sistema de
equações por escalares \(u \in \mathbb{R}^m\):

\begin{equation}
u^TAx = u^Tb
\end{equation}

Com essa nova expressão, podemos modificar algumas características sem a tornar inválida.
Podemos substituir \(=\) por \(\geq\).

\begin{equation}
u^TAx \geq u^Tb
\end{equation}

Feito isso, podemos reduzir o valor de \(u^Tb\) o quanto quisermos, ou seja, sempre existe
um \(c_0 \leq u^Tb\). Como \(x \geq \overrightarrow{\mathrm 0}\), podemos aumentar os valores dos
coeficientes de \(x\), fazendo com que o valor aumente apenas na esquerda, mas a expressão
continue válida.

Sabendo que \(u^TA\) são coeficientes para \(x\), podemos limita-los superiormente pelos
coeficientes do problema original: \(c^T \geq u^TA\). Por fim, temos as seguintes três expressões, todas
válidas:

\begin{equation}
  u^TAx = u^Tb
\end{equation}

\begin{equation}
  c^Tx \geq u^TAx
\end{equation}

\begin{equation}
  u^Tb \geq c_0
\end{equation}

Que quando combinadas:

\begin{equation}
c^Tx \geq u^TAx = u^Tb \geq c_0
\end{equation}

Com isso podemos montar um outro problema, encontrar o valor máximo de \(u\) de forma que \(u^Tb\)
nos sirva como o limite inferior \(c_0\) que buscamos, mas para isso temos que \(c^T \geq u^TA \).

\begin{equation}
\underset{u}{\mathrm max}\{u^Tb\ |\ u \in \{c^T \geq u^TA \} \}
\end{equation}

Assim temos nosso problema dual, que satisfaz a condição que o mínimo do primal é o máximo do
dual pois, por construção, no momento em que multiplicamos o sistema \(Ax = b\) por \(u^T\),
vemos que existem soluções para \(u^TA = c^T\). Portanto podemos dizer que:

\begin{equation}
  c^Tx = u^TAx \implies c^Tx = u^Tb \geq c_0
\end{equation}

Mas como estamos considerando \(u^Tb = c_0\):

\begin{equation}
 c^Tx = u^Tb = c_0 \implies c^Tx = c_0
\end{equation}


Então mostramos que:

\begin{equation}
\underset{x}{\mathrm{min}}\{c^Tx\ |\ x \in P \}
=
\underset{c_0}{\mathrm{max}}\{c_0\ |\ c^Tx \geq c_0\ \forall x \in P\}
=
\underset{u}{\mathrm{max}}\{u^Tb\ |\ u \in \{c^T \geq u^TA \} \}
\end{equation}

Que é o mesmo que dizer que:

\begin{equation}
\underset{x}{\mathrm{min}} = \underset{c_0}{\mathrm{max}} = \underset{u}{\mathrm{max}} \implies c_0 = u^Tb
\end{equation}

Então temos mostrado que a solução do problema primal e dual são iguais (para o caso linear). Esse fato se
torna útil verificação a resolução dos problemas em aplicações praticas. Outra utilidade é que como a
solução do problema dual relaciona todas as restrições à solução do problema primal, isso nos diz algo
sobre os multiplicadores de Lagrange, em alguns casos são usados como estimativas dos multiplicadores.