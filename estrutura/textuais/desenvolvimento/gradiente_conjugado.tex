\section{Método do gradiente conjugado}
TODO: prologo

\subsection{Método do gradiente descendente}
O método do gradiente descendente é um dos métodos iterativos mais simples
que podemos encontrar. Ele consiste em mover o ponto da iteração atual de
acordo com a direção de declive mais acentuada, ou seja, a direção do
gradiente negativo. Considerando que queremos minimizar uma função
\(f: \mathbb{R}^n \mapsto \mathbb{R}^m \), o passo de atualização de um
ponto \( x_k \in \mathbb{R}^n \) é dada por:

\begin{equation}
x_{k+1} = x_k - \alpha_k \nabla f(x_k)
\end{equation}
\( \alpha_k \) nada mais é que um escalar que controla o tamanho da atualização
do ponto atual \(x_k\) na direção de descida. Existem várias possibilidades para
\(\alpha_k\), podemos escolher um valor constante, ou fazer uma estimativa de um
bom valor, ou ainda minimizar também esse valor afim de obter o tamanho ótimo do
passo:

\vspace{-15pt}
\begin{mini!}
{\alpha_k}{ f(x_k - \alpha_k \nabla f(x_k)) \label{gd_obj}}{\label{prob_gd}}{}
\end{mini!}

O gradiente descendente, por mais simples e prático que seja, tem seu preço.
Se pararmos para observar seu comportamento, percebemos um esforço gasto à toa.
Analisando o movimento do ponto atualizado vemos um "zig-zag", isso quer dizer que
a cada iteração, estamos desfazendo um pouco do progresso feito em uma iteração
anterior. Além disso, o gradiente negativo não aponta numa direção pro mínimo, mas
sim nos dá uma direção de minimização imediata em relação ao ponto atual. Se
considerarmos minimizar uma função quadrática convexa com seções elípticas com
alta excentricidade usando o gradiente descendente, depois de muitas iterações
teremos feito muito pouco progresso, visto que o gradiente negativo vai apenas
"rebater" o ponto entre os limites da elipse que são mais "estreitos".

Uma forma de resolver ambos os problemas, é, no lugar de considerar a direção
apenas o gradiente negativo, fazendo com que o gradiente seja ajustado a cada
iteração para ser ortogonal às direções previamente encontradas, o que impede
a perda de progresso pelo comportamento de "zig-zag". Por serem ortogonais,
eles formam um span (base) no \( \mathbb{R}^n \) possibilitando que a solução
seja escrita como uma combinação linear desse span, permitindo a cada iteração
o progresso em apenas uma direção.

\subsection{Construindo o método do gradiente conjugado}

Uma das motivações do método do gradiente conjugado \cite{bonnans2006numerical}
\cite{1987polyak} é resolver um sistema linear de \(n\) equações na forma \(Ax = b \).
Os métodos convencionais para esse tipo de problema falham no sentido de que, para um
\(n\) muito grande, é inviável ter memória para ser guardada uma segunda matriz de
mesmo tamanho, que, caso esparsa, gera um gasto de memória grande demais para pouca
informação.

Considerando \(A\) uma matriz simétrica e definida positiva, podemos definir um produto
interno com respeito à ela:

\begin{equation}
\langle u, v \rangle_A = \langle u, Av \rangle =  \langle Au, v \rangle 
\end{equation}

Onde dizemos que os vetores \(u\) e \(v\) são conjugados em respeito a \(A\) (ou A-conjugados) se:

\begin{equation}
\langle u, v \rangle_A = 0
\end{equation}

A partir dai podemos escolher um conjunto de \(n\) vetores que formam um span do
\( \mathbb{R}^n \):

\begin{equation}
U := \{d_1, d_2, ..., d_n\}
\end{equation}

\begin{equation}
\label{gc_prod_interno}
\langle d_i, d_j \rangle_A = 0; i \ne j
\end{equation}

Podemos escrever qualquer vetor \(v \in \mathbb{R}^n\) como uma combinação linear dos
vetores que formar \(U\):

\begin{equation}
v = \sum_{i=1}^n \alpha_i d_i
\end{equation}

Ou seja, podemos escrever a solução de \(Ax = b\) dessa forma. Chamemos a solução de
\(x^*\). Temos então:

\begin{equation}
x^* = \sum_{i=1}^n \alpha_i d_i
\end{equation}

Multiplicando A pela esquerda nos dois lados e depois multiplicando por algum \(d_k\):

\begin{equation}
Ax^* = \sum_{i=1}^n \alpha_i Ad_i \implies  d_k A x^* = \sum_{i=1}^n \alpha_i d_k A d_i
\end{equation}

Substituindo \(Ax^*\) por um vetor \(b\) e explicitando o produto interno no lado direito:

\begin{equation}
  \langle d_k, b \rangle = \sum_{i=1}^n \alpha_i d_k A d_i \implies \langle d_k, b \rangle = \sum_{i=1}^n \alpha_i \langle d_k ,d_i \rangle_A  
\end{equation}

Considerando \ref{gc_prod_interno}, apenas a parcela \( \langle d_k, d_k \rangle_A \) não é nula, e rearranjando para
\(\alpha_k\):

\begin{equation}
\langle d_k, b \rangle  = \alpha_k \langle d_k , d_k \rangle_A \implies \frac{\langle d_k, b \rangle}{\langle d_k , d_k \rangle_A} = \alpha_k
\end{equation}

Logo, para calcular a solução diretamente, basta encontrar o conjunto \(U\).

Antes de podermos encontrar o tal conjunto, primeiro vamos observar uma outra forma de
interpretar o problema. Então, podemos escrever um problema do tipo \(Ax = b\) como um
problema de otimização de uma função quadrática convexa:

\vspace{-15pt}
\begin{mini!}
{x}{ f(x) = \frac{1}{2} \langle x, x \rangle_A - \langle b, x \rangle \label{gcq_obj}}{\label{prob_gcq}}{}
\end{mini!}

Como a função é quadrática e convexa, sabemos que possui apenas um único ponto de
extremo \(x^*\) e que é o mínimo. A partir dessas informações podemos definir as
seguintes expressões:

\begin{equation}
\nabla f(x^*) = 0
\end{equation}

Para termos \(\nabla f(x)\), podemos simplificar o processo de derivação de \(f(x)\) ao imaginar
que \(\langle x, x \rangle_A\) é de alguma forma equivalente à \(x^2\), e \(\langle x, b \rangle\)
ao produto de uma constante e a variável real. Portanto:

\begin{equation}
\nabla f(x) = Ax - b
\end{equation}

E da mesma forma:

\begin{equation*}
\nabla^2 f(x) = A
\end{equation*}

Com isso podemos usar o mesmo método para encontrar soluções para \(Ax = b\) para
minimizar funções cuja matriz hessiana (\(\nabla^2 f(x)\)), que chamaremos sempre de \(A\),
seja definida positiva.

Podemos gerar o conjunto \(U\) a partir dos gradientes, ajustando-os para que sejam
mutualmente conjugados. Escolhendo dessa forma, pode não ser necessário ter-se todos os
vetores que formam \(U\), já que como a combinação linear de alguns poucos gradientes
encontrados durante as iterações do gradiente descendente nos dão boas aproximações da
solução, é esperado que esses vetores de descida corrigidos deem ainda uma melhor
solução.

Comecemos fazendo com que \(d_1 = -\nabla f(x_1) = -g_1 \), já que isso pode nos por na
direção correta do mínimo. O algoritmo, sendo iterativo, tem uma forma semelhante ao
gradiente descendente, mudando apenas a atualização:

\begin{equation}
\label{att_gc}
x_{k+1} = x_k + \alpha_k d_k
\end{equation}

Onde \(d_k \in U\) e \(\alpha_k\) é o tamanho do passo, que pode ser exato minimizando
\(f\) na direção \(d_k\).

Como queremos direções conjugadas, que por sua vez são geradas a partir dos gradientes,
então o gradiente no próximo ponto já pode ser conjugado em relação à direção atual.
Considerando \ref{att_gc}, podemos dizer que:

\begin{equation}
  \label{def_gc_gk1}
  \nabla f(x_{k+1}) = g_{k+1} = g_k + \alpha_k Ad_k
\end{equation}

Com isso, queremos que:

\begin{equation}
\label{def_gc_dk_gk1}
\langle g_{k+1}, d_k \rangle = 0 \implies \langle g_k + \alpha_k A d_k, d_k \rangle = 0
\end{equation}


Considerando as propriedades inerentes a um produto interno:

\begin{equation}
\langle g_k, d_k \rangle + \alpha_k \langle A d_k, d_k \rangle = 0 \implies \alpha_k \langle A d_k, d_k \rangle = - \langle g_k, d_k \rangle
\end{equation}

Por fim, resolvendo para \(\alpha_k\) e mudando a notação:

\begin{equation}
\alpha_k= - \frac{\langle g_k, d_k \rangle}{\langle A d_k, d_k \rangle} \implies \alpha_k= - \frac{\langle g_k, d_k \rangle}{\langle d_k, d_k \rangle_A}
\end{equation}


O que nos falta agora é a atualização de \(d_k\). Como dito anteriormente, queremos
construir essas direções a partir do gradiente negativo corrigido para que seja
A-ortogonal em relação as direções já encontradas.

\begin{equation}
\label{def_dk1}
d_{k+1} = -g_{k+1} + \beta_k d_k
\end{equation}

Portando, \(\beta_k\) deve ser escolhido de forma que \(\langle d_{k+1}, d_k \rangle_A = 0\):

\begin{equation}
\langle -g_{k+1} + \beta_k d_k, d_k \rangle_A = 0 \implies \langle -g_{k+1}, d_k \rangle_A + \beta_k \langle d_k, d_k \rangle_A = 0
\end{equation}

Rearranjando e resolvendo para \( \beta_k \):

\begin{equation}
  \beta_k \langle d_k, d_k \rangle_A = - \langle -g_{k+1}, d_k \rangle_A \implies \beta_k = -\frac{\langle -g_{k+1}, d_k \rangle_A}{\langle d_k, d_k \rangle_A}
\end{equation}

Já temos o algoritmo pronto, mas ainda é custoso. Com mais um pouco de trabalho, podemos
remover duas multiplicações por A de \(\beta_k\):

\begin{equation}
\label{def_beta_gc}
\beta_k = - \frac{\langle -g_{k+1}, Ad_k \rangle}{\langle d_k, Ad_k \rangle}
\end{equation}

Queremos substituir \(Ad_k\) por algo mais simples, para isso podemos usar \ref{def_gc_gk1}:

\begin{equation}
\label{def_adk}
g_{k+1} = g_k + \alpha_k A d_k \implies g_{k+1} - g_k = \alpha_k A d_k \implies A d_k = \frac{g_{k+1} - g_k}{\alpha_k}
\end{equation}

Como \(\beta_k\) é definido por uma razão, podemos trabalhar o numerador e o denominador
separadamente. Então comecemos trabalhando apenas o numerador de \(\beta_k\):

\begin{equation}
\langle -g_{k+1}, Ad_k \rangle
\end{equation}

Substituindo \(Ad_k\) pela definição em \ref{def_adk}, e extraindo os coeficientes:

\begin{equation}
\langle -g_{k+1}, \frac{g_{k+1} - g_k}{\alpha_k} \rangle \implies -\frac{1}{\alpha_k} \langle g_{k+1}, g_{k+1} - g_k \rangle
\end{equation}

E pela linearidade do produto interno:

\begin{equation}
-\frac{1}{\alpha_k}(\langle g_{k+1}, g_{k+1} \rangle + \langle -g_{k+1}, - g_k \rangle)
\end{equation}


Sabemos, porque construímos assim, que \(\langle g_{k+1}, g_k \rangle = 0\). Por fim, temos que o
numerador de \(\beta_k\) é dado por:

\begin{equation}
\label{def_gc_numerador}  
-\frac{1}{\alpha_k} \langle g_{k+1}, g_{k+1} \rangle
\end{equation}

Agora trabalhando o denominador de \(\beta_k\):

\begin{equation}
\langle d_k, Ad_k \rangle
\end{equation}


Novamente, usando \ref{def_adk}, e extraindo o coeficiente:

\begin{equation}
\langle d_k, \frac{g_{k+1} - g_k}{\alpha_k} \rangle \implies \frac{1}{\alpha_k}\langle d_k, g_{k+1} - g_k \rangle
\end{equation}

Pela linearidade do produto interno e usando \ref{def_gc_dk_gk1}:

\begin{equation*}
\frac{1}{\alpha_k} (\langle d_k, g_{k+1} \rangle + \langle d_k, - g_k \rangle) \implies \frac{1}{\alpha_k} \langle d_k, - g_k \rangle
\end{equation*}

Considerando \ref{def_dk1} para substituir \(d_k\):

\begin{equation}
\frac{1}{\alpha_k} \langle -g_{k} + \beta_{k-1} d_{k-1}, - g_k \rangle
\end{equation}

Aplicando a linearidade e extraindo coeficientes:


\begin{equation}
\frac{1}{\alpha_k}(
\langle -g_{k}, - g_k \rangle +
\langle \beta_{k-1} d_{k-1}, - g_k \rangle
) \implies
\frac{1}{\alpha_k}(
\langle -g_{k}, - g_k \rangle -
\beta_{k-1}(\langle d_{k-1}, g_k \rangle)
)
\end{equation}

Por fim, considerando \ref{def_gc_dk_gk1}, temos o denominador dado por:

\begin{equation}
\label{def_gc_denominador}
\frac{1}{\alpha_k} \langle -g_{k}, - g_k \rangle
\end{equation}


Finalmente, podemos reescrever \(\beta_k\). Considerando \ref{def_beta_gc}, \ref{def_gc_numerador} e \ref{def_gc_denominador}, e substituindo:

\begin{equation}
\beta_k = - \frac{\langle -g_{k+1}, Ad_k \rangle}{\langle d_k, Ad_k \rangle} \implies \beta_k = - \frac{-\frac{1}{\alpha_k} \langle g_{k+1}, g_{k+1} \rangle}{\frac{1}{\alpha_k} \langle -g_{k}, - g_k \rangle}
\end{equation}

Isolando e resolvendo o coeficiente:

\begin{equation}
  \beta_k = - \frac{-\frac{1}{\alpha_k}}{\frac{1}{\alpha_k}} \cdot \frac{\langle g_{k+1}, g_{k+1} \rangle}{\langle -g_{k}, - g_k \rangle}   \implies \beta_k = - (-1) \frac{\langle g_{k+1}, g_{k+1} \rangle}{\langle -g_{k}, - g_k \rangle}
\end{equation}

Por fim, temos a ultima peça para a definição do algoritmo do gradiente conjugado:

\begin{equation}
\beta_k = \frac{\langle g_{k+1}, g_{k+1} \rangle}{\langle -g_{k}, - g_k \rangle}
\end{equation}

Em resumo, iniciando com \(x_1\) sendo uma estimativa do ótimo e a direção inicial sendo \(d_1 = -\nabla f(x_1) = -g_1\), temos que o algoritmo é:

\begin{equation}
x_{k+1} = x_k + \alpha_k d_k
\end{equation}

\begin{equation}
\alpha_k= - \frac{\langle g_k, d_k \rangle}{\langle d_k, d_k \rangle_A}
\end{equation}

\begin{equation}
d_{k+1} = -g_{k+1} + \beta_k d_k
\end{equation}

\begin{equation}
\beta_k = \frac{\langle g_{k+1}, g_{k+1} \rangle}{\langle -g_{k}, - g_k \rangle}
\end{equation}

Podemos ver que o método não é complexo de ser usado, nem tão custoso computacionalmente em casos simples.
Em problemas de maior dimensão, funções custosas ou ainda problemas difíceis que não seja quadráticos, o
método pode não ser o recomendado pelo fato de considerar varias avaliações da função para uma única iteração.