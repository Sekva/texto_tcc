\section{O modernismo sob o ponto de vista da programação}
Ao observarmos o percusso histórico dos métodos de otimização, podemos ir tão atrás no tempo
quanto a matemática é usada, mas para que possamos considerar passagens de eras é necessário
buscar acontecimentos com uma certa relevância, e que tais acontecimentos estejam relacionados
a contextos de relevância na época.

Um grande acontecimento que podemos tomar como algo que marque a passagem de uma era para
outra é o desenvolvimento do cálculo por Newton (sabe-se que existem outros, mas Newton é
o mais creditado). Ao observarmos o que seria um processo de otimização matemática prévio
ao cálculo, vemos que parece ser algo de certa forma não tão eficiente ou prático. Mesmo
considerando um problema simples, tomando que sua interpretação por meio de gráficos bastaria
para uma resolução do problema, não seria algo tão trivial como percebemos hoje, uma vez
que até a própria estruturação de problemas gráficos era algo recente no momento de Newton,
já que a estruturação de coordenadas do plano acabava de ser feita por Descartes.

Logo após o desenvolvimento do cálculo, toda uma nova forma de se considerar problemas
que já existiam ou ainda derivar novos problemas foi conhecida. Nomes como Bernoulli, Euler,
Cauchy (\ref{secao_historia}) Weierstrass (\ref{Teorema_de_Weierstrass}) comprovam a rápida
aceitação das ideias de cálculo e análise matemática, como também formam bases sólidas para
o que temos sobre otimização atualmente. Métodos já tinham sido propostos, bem como algumas
de suas propriedades, como o próprio método de Newton (em colaboração com Raphson), ou ainda
a primeira aparição do método do gradiente descendente, por Cauchy.

Desde essa época, o estudo da otimização foi tomando ramos e se especializando cada vez mais.
As especializações normalmente surgem de acordo com modelos de problemas reais que deseja-se
otimizar, por isso hoje somos capazes de encontrar toda uma vasta bibliografia a respeito de
otimização de problemas lineares, não lineares, considerando variáveis inteiras, variáveis
reais, ou ambas.

Ao considerarmos os interesses de cada época em aprimorar o conhecimento matemático, podemos
perceber acontecimentos que estão relacionados a esses interesses. Ao tomarmos como um ponto
de partida o movimento iluminista no século XVIII, vemos e sabemos a relação com os estudos
de Newton e os mesmos de sua época, e também compreendemos que o ideal iluminista contribuiu
bastante para a disseminação desses estudos. Um segundo acontecimento que podemos levar em
consideração é a revolução industrial. Conceitos físicos de máquinas pensadas durante a
revolução industrial são construídos sobre a nova fundação do cálculo, permitindo, dentre
outras coisas, melhor eficiência das máquinas. Podemos dizer que um dos pontos altos desse
período, mesmo que tardio, é o trabalho de Maxwell sobre a termodinâmica, o qual, de forma
geral, é baseado em conceitos de estabilidade de sistemas.


O movimento de industrialização sempre crescente começou a demonstrar seus pontos fracos,
principalmente em relação às eficiências das indústrias, ainda no começo do século XX,
quando as demandas cresceram, logísticas sobre recursos de produção foram dificultadas,
e tensões internacionais estavam se formando. Passada a Primeira Guerra Mundial, diversos
países se encontram com recursos escassos e buscando mais atenção no gerenciamento desses
recursos. Essa atenção nos leva à dois nomes que desenvolveram, de certa forma, as mesmas
ferramentas para as mesmas finalidades, Kantorovich(\ref{sec_kantorovich}) e Dantzig
(\ref{sec_dantzig}), ambos buscando otimizar o processo de alocação de pessoas, recursos,
ou algo em relação com o potencial de produção. Mesmo havendo um interesse maior em
otimização linear nesse período, outras áreas da otimização vinham sendo desenvolvidas
também, um exemplo disso são as condições KKT (\ref{sec_kkt}).

Agora após a Segunda Guerra Mundial, os computadores digitais começam a ganhar mais espaço.
Na década de 50, como uma das primeiras linguagens de programação, temos o Fortran, do
inglês Formula Translation, como sendo uma linguagem pensada para os cálculos matemáticos
precisos e custosos de se fazer manualmente. Junto a isso, o recém desenvolvimento da otimização
linear e as novas bases formadas para outras áreas da otimização expandiu o interesse que
se tinha na época.

Quando chegamos na década de 60, encontramos os nomes e acontecimentos já citados na seção
\ref{secao_historia}, os quais agora podem ser compreendidos como em uma era moderna. E
seguindo daí, não é mais tão claro um acontecimento com influência na otimização para
podermos separar uma nova era.

Olhando agora para o lado dos computadores, enquanto a otimização já estava em seu
período moderno, os computadores estavam apenas começando, sendo usados mainframes
para cálculos já computacionalmente desenvolvidos, o que acabava restringindo o
acesso ao estudo nesse meio, mesmo considerando o interesse crescente no assunto.
Considerando os avanços rápidos, Fortran nunca deixou de ser usada e atualizada,
um exemplo é que ainda hoje diversas bibliotecas matemáticas em fortran são usadas.
Avanços, como invenção e difusão de linguagens estruturadas (C e Ada) contribuíram
na difusão e acessibilidade dos estudos sobre otimização, tanto que o DONLP2
(\ref{sec_dosnlp2}) que foi escrito em C, e Ada é uma linguagem usada até hoje
como linguagem para sistemas críticos, como sistemas operacionais da aviação.

Agora considerando como um todo a análise histórica apresentada no capítulo
\ref{chap:fundamentacaoTeorica}, o SCP se encaixa em um contexto com um ideal
semelhante ao renascentismo. É dito na apresentação do SCP que um de seus objetivos
é recuperar o interesse no uso de técnicas de resolução de subproblemas lineares
que haviam sido deixados de lado em favor dos métodos quadráticos. Considerando
os resultados obtidos pelos autores do SCP, foi mostrado por eles que houve
um avanço para outras técnicas de otimização sem antes esgotar as possibilidades
do que os problemas lineares tinham a oferecer. Podemos ver isso ao considerar
que as técnicas empregadas no SCP já existiam, ou pelo menos tinham suas bases
firmes, no momento que os métodos quadráticos ganharam sua fama. Dito isso, podemos
colocar o SCP como sendo um método pós-moderno/renascentista, também considerando
a década de sua concepção, de 2000 à 2010.
